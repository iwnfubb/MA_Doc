
\begin{abstract}
Bei meiner Masterarbeit geht es um eine Echtzeitanwendung zur Erkennung von au�ergew�hnlichen Situationen f�r alte Menschen, die allein lebend sind. Die Eingaben des Programms sind Bilder von einer Bosch 360-Grad-Kamera. Durch ein Verfahren von Hintergrundsubtraktion wird die Bewegung einer Person als eine schwarzwei�e Bild erstellt. Nach dem Segmentierung des Vordergrundes wird eine Silhouette  der Person kreiert. Mit Hilfe von einer Histogramm Analyse auf Silhouette wird eine K�rperhaltung der Person gesch�tzt. Die K�rperhaltung kann Stehen, Liegen, Sitzen oder Beugen sein. Anhand Positionen wo die Person oft bewegt und wann sie steht, sitzt oder liegt... k�nnen au�ergew�hnlichen Situationen mithilfe Fuzzylogik erkannt werden. Die Bosch 360-Grad-Kamera hat eine gro�artige F�higkeit, in die Richtung der Bewegung zu blicken. Das Feature wird auch in diesem Projekt genutzt, damit die au�ergew�hnliche Situationen im ganzen Raum betrachtet werden kann.\\
Nach vollst�ndiger Optimierung und vielen Tests betrugt die Verarbeitungszeit f�r eine Durchgang 100 Millisekunden. Bei Segmentierung der Silhouette gibt das Programm fast genau Silhouette wie die Erwartung zur�ck. Die Sch�tzung einer K�rperhaltung betrugt auch mehr als $80\%$ der Genauigkeit. Die Anwendung des Projektes liefert zuverl�ssige Ergebnisse mit alle richtige Erkennung au�ergew�hnlicher Situation zur�ck.  
\end{abstract}
 
\newpage\thispagestyle{empty}\hspace{1em}\newpage

\begin{otherlanguage}{english}
\begin{abstract}
...
(exakte englische �bersetzung der deutschen Kurzfassung)
\end{abstract}
\end{otherlanguage} 

\newpage\thispagestyle{empty}\hspace{1em}\newpage