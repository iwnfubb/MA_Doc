\begin{abstract}
In dieser Masterarbeit geht es um eine Echtzeitanwendung, die erkennt, ob ein Unfall passiert ist. Dabei zielt diese Software auf alte Menschen ab, die zu Hause wohnen. Die Anwendung wertet Bildmaterial aus, das mit einer 360� Kamera aufgenommen wird. Durch das Hintergrundsubtraktion-Verfahren werden sich bewegende Objekte erkannt und als Bin�rmasken dargestellt. Im Prinzip funktioniert die Hintergrundsubtraktion durch die Analyse der Pixel. Die Pixel, die statisch und lange unver�ndert in der Szene bleiben, werden als schwarzer Hintergrund markiert. Sich bewegende Objekte werden als Silhouetten bestehend aus wei�en Pixel kodiert. Dann wird die K�rperhaltung einer Person mithilfe der Histogrammanalyse aus der Silhouette gesch�tzt. Die erkannten K�rperhaltungen sind z.B.: Liegend oder Stehend. Anhand von Positionen, wo sich die Person h�ufig aufh�lt oder bewegt, k�nnen dann au�ergew�hnliche Situationen mithilfe der Fuzzylogik erkannt werden. Die Fuzzylogik ist ein Ansatz zum Berechnen der Wahrscheinlichkeit, dass ein Ereignis durch bestimmte Bedingungen eintritt. In dieser Arbeit hilft diese Logik bei der Erkennung, ob eine Situation durch K�rperhaltung, Zeitraum und h�ufige Position normal oder abnormal ist. Zus�tzlich ist die 360� Kamera mit einem Bewegungssensor ausgestattet und kann Objekten folgen, die sich bewegen. Diese F�higkeit wird genutzt, damit au�ergew�hnliche Situationen im ganzen Raum beobachtet werden k�nnen.\\
Die Zeit f�r die Berechnung eines Bildes bis zur Generierung einer Bewertung und die Entscheidung, ob es sich um einen abnormalen Fall handelt oder nicht, betrug im Schnitt 59 Millisekunden. Es wurden mehrere Videos als Eingabe f�r die Echtzeitanwendung verwendet. Auf den Videos sind Menschen in einem Raum zu sehen, die in verschiedenen Situationen sind. Dabei wurden eine Reihe von au�ergew�hnlichen Situationen simuliert und �berpr�ft, ob die entwickelte Software diese auch erkennt. Das Programm hat dabei alle au�ergew�hnlichen Situationen erkannt. Die richtige Klassifizierung jedes Bildes in normal oder abnormal war dabei nicht immer korrekt. Es werden dennoch alle abnormalen Situationen richtig klassifiziert, da der verwendete Algorithmus nicht ausschlie�lich von einzelnen Bildern abh�ngt.
\end{abstract}

\newpage

\begin{otherlanguage}{english}
\begin{abstract}
In this master's thesis a real-time application was developed to recognize accidents. This application is targeted at elder people who live at home. The program evaluates incoming images recorded by a 360� camera. The background subtraction creates a binary image with which moving objects can be detected and revealed. In principle, the background subtraction works on analyzing the pixels. Pixels which are static and not changing within the scene are marked to be the black background. Moving objects are silhouettes which are encoded as white pixels. Then, the posture of a person can be determined by the Histrogram analysis of the silhouette. For example, the recognized postures can be Laying or Standing. Based on positions, where a person is staying or moving for a long period of time, the fuzzy logic can detect abnormal situations. Fuzzy logic is further used for computing the probability of the occurrence of a given event, based on specific conditions. In this thesis, this kind of logic was used to predict the situation, depending on calculated postures, time span and frequent positions. In addition, the 360� camera can follow moving object by using its on-board motion detector. This ability is used to determine abnormal situations in the whole room.\\
On average, it took 59 milliseconds for the computation of one image to decide whether the situation is normal or abnormal. Various videos with people shown in a room in different situations were used as input for the software. In these videos, many abnormal situations were simulated in order to test, if the developed real-time application will recognize them. The application could detect all abnormal situations. However, the classification of each image was not always correct. But all abnormal situations were still being recognized, because the used algorithm is not exclusively dependent on single images.
\end{abstract}
\end{otherlanguage} 

\newpage
