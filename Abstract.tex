
\begin{abstract}
In dieser Masterarbeit geht es um eine Echtzeitanwendung, die zur Erkennung von au�ergew�hnlichen Situationen wie z.B.: Das Umfallen einer �lteren Person.
Die Anwendung wertet Bildmaterial aus, das mit einer 360� Kamera aufgenommen wird. Durch das Hintergrundsubtraktions-Verfahren wird die bewegte Person dann in ein Schwarzwei�bild dargestellt. Die nach der Segmentierung des Vordergrunds erzeugte Silhouette, stellt die bewegte Person dar. Die K�rperhaltung einer Person wird mithilfe einer Histogrammanalyse aus der Silhouette gesch�tzt. Die erkannten K�rperhaltungen sind z.B.: Liegend oder Stehend. Anhand von Positionen, wo sich die Person h�ufig aufh�lt oder bewegt, k�nnen dann au�ergew�hnliche Situationen mithilfe der Fuzzylogik erkannt werden. Au�erdem ist die 360� Kamera mit einem Bewegungssensor ausgestattet und folgt Objekte, die sich bewegen.\\
Das Segmentierungs-Verfahren der Silhouette liefert mit sehr geringen Abweichungen, das Ergebnis des \glqq{}Ground True\grqq{}.
Die Genauigkeit der K�rperhaltungssch�tzung betr�gt $80\%$. Die Echtzeitanwendung liefert somit zuverl�ssige Ergebnisse aus der Erkennung von au�ergew�hnlichen Situationen.
\end{abstract}
 
\newpage\thispagestyle{empty}\hspace{1em}\newpage

\begin{otherlanguage}{english}
\begin{abstract}
...
(exakte englische �bersetzung der deutschen Kurzfassung)
\end{abstract}
\end{otherlanguage} 

\newpage\thispagestyle{empty}\hspace{1em}\newpage
