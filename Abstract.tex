
\begin{abstract}
In dieser Masterarbeit geht es um eine Echtzeitanwendung, die zur Erkennung von außergewöhnlichen Situationen wie z.B.: Das Umfallen einer älteren Person.
Die Anwendung wertet Bildmaterial aus, das mit einer 360° Kamera aufgenommen wird. Durch das Hintergrundsubtraktions-Verfahren wird die bewegte Person dann in ein Schwarzweißbild dargestellt. Die nach der Segmentierung des Vordergrunds erzeugte Silhouette, stellt die bewegte Person dar. Die Körperhaltung einer Person wird mithilfe einer Histogrammanalyse aus der Silhouette geschätzt. Die erkannten Körperhaltungen sind z.B.: Liegend oder Stehend. Anhand von Positionen, wo sich die Person häufig aufhält oder bewegt, können dann außergewöhnliche Situationen mithilfe der Fuzzylogik erkannt werden. Außerdem ist die 360° Kamera mit einem Bewegungssensor ausgestattet und folgt Objekte, die sich bewegen.\\
Das Segmentierungs-Verfahren der Silhouette liefert mit sehr geringen Abweichungen, das Ergebnis des \glqq{}Ground True\grqq{}.
Die Genauigkeit der Körperhaltungsschätzung beträgt $80\%$. Die Echtzeitanwendung liefert somit zuverlässige Ergebnisse aus der Erkennung von außergewöhnlichen Situationen.
\end{abstract}
 
\newpage\thispagestyle{empty}\hspace{1em}\newpage

\begin{otherlanguage}{english}
\begin{abstract}
...
(exakte englische Übersetzung der deutschen Kurzfassung)
\end{abstract}
\end{otherlanguage} 

\newpage\thispagestyle{empty}\hspace{1em}\newpage
