
\begin{abstract}
In dieser Masterarbeit geht es um eine Echtzeitanwendung, die zur Erkennung von au�ergew�hnlichen Situationen wie z.B.: Das Umfallen einer �lteren Person. Das Ziel der Masterarbeit ist Erkennung, ob die beobachtete Person gut geht oder die Person nach einem Unfall umf�llt. Die Anwendung wertet Bildmaterial aus, das mit einer 360� Kamera aufgenommen wird. Durch das Hintergrundsubtraktion-Verfahren wird die bewegte Person dann in eine Bin�rmaske dargestellt. Im Prinzip funktioniert die Hintergrundsubtraktion durch die �nderungen der Pixel in Szene. Die Pixel, die statisch und lang in Szene bleiben, werden als schwarzer Hintergrund markiert, sonst werden die als Vordergrund mit Wei� dargestellt. Die nach der Segmentierung des Vordergrunds erzeugte Silhouette, stellt die bewegte Person dar. Die K�rperhaltung einer Person wird mithilfe einer Histogrammanalyse aus der Silhouette gesch�tzt. Die erkannten K�rperhaltungen sind z.B.: Liegend oder Stehend. Anhand von Positionen, wo sich die Person h�ufig aufh�lt oder bewegt, k�nnen dann au�ergew�hnliche Situationen mithilfe der Fuzzylogik erkannt werden.  Fuzzylogik ist ein Ansatz zum Berechnen die Wahrscheinlichkeit, dass eine Ereignis durch bestimmte Bedingungen passiert. In dieser Arbeit hilft Fuzzylogik bei der Erkennung, ob eine Situation durch K�rperhaltung, Zeitraum und h�ufige Position normal oder abnormal ist, angewendet. Au�erdem ist die 360� Kamera mit einem Bewegungssensor ausgestattet und folgt Objekte, die sich bewegen. Diese F�higkeit wird auch in diesem Projekt genutzt, damit au�ergew�hnliche Situationen im ganzen Raum beobachtet werden k�nnen. \\
Die Bearbeitung eines Bildes bis zum Generierung einer Bewertung, ob ein abnormaler Fall passiert, betrugt 100 Millisekunden. Das Segmentierungs-Verfahren der Silhouette liefert mit sehr geringen Abweichungen, das Ergebnis des \glqq{}Ground True\grqq{}. Die quantitative Evaluation der K�rperhaltung kam zu einer Genauigkeit von 80\%. Auf den selbst aufgenommenen Videosequenzen konnten alle au�ergew�hnlichen Situationen korrekt erkannt werden.
\end{abstract}
 
\newpage\thispagestyle{empty}\hspace{1em}\newpage

\begin{otherlanguage}{english}
\begin{abstract}
...
(exakte englische �bersetzung der deutschen Kurzfassung)
\end{abstract}
\end{otherlanguage} 

\newpage\thispagestyle{empty}\hspace{1em}\newpage
