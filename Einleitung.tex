\chapter{Einleitung}
Diese Masterarbeit beschäftigt sich mit der Erkennung von außergewöhnlichen Situationen, besonders bei älteren Menschen, die alleine leben. Dazu wird zuerst eine Hintergrundsubtraktion angewendet, um das bewegte Objekt zu erfassen. Das dadurch erzeugte Binärbild wird mit dem Segmentierung-Verfahren in verschiedenen \acs{ROI} Bildern geschnitten, um eine Silhouette des bewegten Objekts zu erzeugen. Dann wird eine Histogrammanalyse angewendet, um die Körperhaltung der Person aus der Silhouette zu bestimmen. Eine Hintergrunddichte-Rechnung wird gemacht, um Drehbewegungen der Kamera zu erkennen. Bei Bewegungen der Kamera, werden die Hintergrundsubtraktion Daten aktualisiert. Wahrscheinlichkeitsberechnungen kommen letztlich zum Einsatz, um die abnormalen Situationen zu bewerten. Die OpenCV Bibliothek bietet effiziente Lösungen zur Hintergrundsubtraktion und ist mittels einer Open-Source-Lizenz veröffentlicht. Somit ist OpenCV Bibliothek ideal für diese Art von Bildverarbeitungen und wird in diesem Projekt eingesetzt.

%%%%%%%%%%%%%%%%%%%%%%%%%%%%%%%%%%%%%%%%%%%%%%%%%%%%%%%%%%%%%%%%%%%%%%%%%%%%%%%%%%%
\section{Motivation}
Sicherheitssysteme werden durch rapiden technologischen Weiterentwicklungen immer zuverlässiger. Sie finden in immer mehr Bereichen des Lebens Einzug. Zum Beispiel kommt das neue eCall Sicherheitssystem von der eSafty-Initiative bei vielen Kraftfahrzeugen zum Einsatz. In der Europäischen Union wird dieses Sicherheitssystem ab dem 31. März 2018 für alle neuen Kraftfahrzeuge sogar zur Pflicht. Kraftfahrzeuge mit eingebautem eCall erkennen automatisch einen Verkehrsunfall und melden diesen über das einheitliche System zur Meldung eines Verkehrsunfall. Auch in anderen Bereichen wie in Wohnungen siedelt sich immer mehr Smarte-Technik an. Durch den Einzug von Smart-Systemen in den Wohnungen, kommen auch mehr Sicherheitssysteme mit. Sprachgesteuerte Systeme wie Google Home und Amazon Alexa sind schon fast standard. Dazu kommen etliche Erweiterungen wie z.B.: Eine Kamera, um Skype anzusteuern und eine Video-Konferenz zu starten. Die Kamera kann dann auch der Person folgen, wenn diese sich im Raum bewegt und bietet somit sehr flexible Kommunikationsmöglichkeiten. Das ähnlicht das Prinzip des Smart-Home. Smart-Home ist ein großer Begriff, der im wesentlichen bedeutet, dass das Zuhause über Sprachsteuerung oder andere Geräte steuerbar ist. Smart-Geräte sind sehr vielfältig. Es können smarte Toaster oder TVs an das Smart-Home angeschlossen werden und somit von der Ferne steuerbar. Auch die dadurch entstandenen Sicherheitssysteme sind aktiv und können aus der Ferne verwaltet werden. Zum Beispiel kann die Kamera, die für Video-Konferenzen genutzt wird, auch aus der Ferne aktiviert werden, um nach dem Rechten zuschauen. Dies ermöglicht auch automatisierte Systeme, das Smart-Home zu schützen, indem es durch die Kamera das Smart-Home überwacht und auf Einbrecher aufmerksam macht. Um solche Systeme zuentwickeln, braucht das Sicherheitssystem eine Personenerkennung. Diese muss mit einer guten Präzision arbeiten, um zum Beispiel Haustiere wie Hunde und Katzen nicht als Einbrecher zu kennen. Jedoch können auch andere Sicherheitssysteme hier zum Einsatz kommen, die gezielt ältere Menschen schützt. So könnte das System ältere Menschen bei einem Umfallen erkennen und Hilfe anfordern. Besonders ältere Menschen sind durch das Hinfallen sehr gefährdet. Solche System verbessern die Sicherheit dieser Menschen, die sonst bei Unfällen hilflos wären und keinen Notruf selbst tätigen können. Viele Smart-Home Wohnungen besitzen bereits Kameras oder sogar fortschrittliche 360° Kameras. Die Motivation ist es hier anzusetzen und davon Gebrauch zu machen, um besonders das Leben von gefährdeten Menschen sicherer zu machen.

%%%%%%%%%%%%%%%%%%%%%%%%%%%%%%%%%%%%%%%%%%%%%%%%%%%%%%%%%%%%%%%%%%%%%%%%%%%%%%%%%%%
\section{Ziele}\label{chp:Ziele}   
Das Ziel dieser Arbeit ist die Entwicklung eines Programms, das mit Hilfe von Kamera Bewegungen erkennt und dadurch Vordergrund von Hintergrund extrahiert und analysiert. Am Ende soll das Programm die Situationen abschätzen, ob es ein Unfall gibt. Das Programm soll die außergewöhnlichen Situationen richtig und pünktlich erkennen. Die Bildverarbeitungen des Projektes sollen eine gute Segmentierung der Bewegung sowie richtige Körperschätzung der Person bringen. Am Ende soll das Programm zuverlässige Ergebnisse liefern.      
%%%%%%%%%%%%%
\subsubsection{Echtzeit} 
Da es um einen Unfallmelder geht, deshalb muss die Anwendung schon in Echtzeit laufen. Die durchschnittliche Laufzeit des Programms soll mindesten 10 Frames pro Sekunde benötigen, um die Situation in der Echtzeit zu erkennen und die außergewöhnlichen Situationen pünktlich zu detektiert.  

%%%%%%%%%%%%%
\subsubsection{Kosten} 
Das Programm sollte mit der normalen, kostengünstigen Privatanwender-Hardware laufen, sodass für die Ausführung keine Spezialrechner benötigt werden. Die einzige Voraussetzung an das System ist eine Kamera. In diesem Projekt wird die 360-Grad-Kamera von Bosch angewendet.  

%%%%%%%%%%%%%%%%%%%%%%%%%%%%%%%%%%%%%%%%%%%%%%%%%%%%%%%%%%%%%%%%%%%%%%%%%%%%%%%%%%%
\section{Aufbau der Arbeit}
Nächster Abschnitt geht es um verwandte Arbeite, die in Richtung von Hintergrundsubtraktion und Körperhaltungserkennung, repräsentiert.
Im ersten Abschnitt des Kapitels~\ref{chp:Grundlageb} soll zunächst angewendete Hintergrund Subtraktionen verdeutlicht werden. Danach soll im zweiten Abschnitt die Histogrammanalyse beschrieben werden, bevor im Abschnitt~\ref{sec:OpenCV} ein Überblick über das OpenCV Framework angegeben wird. An dieser Stelle werden auch die Vorteile der Anwendung von OpenCV bei dieser Arbeit genauer erläutern. Kapitel~\ref{chp:EigVerfahren} beschäftigt sich mit der Theorie des im Rahmen dieser Masterarbeit entwickelten Algorithmus. Dabei sollen die einzelnen Schritte des Algorithmus genauer erläutert werden. Die Implementierung des Algorithmus wird im diesen Kapitel besprochen, dabei sollen die benutzten Bibliotheken sowie die Umsetzung des Programms beschrieben werden. Anschließend werden im Kapitel 4 die durchgeführten Tests und deren Ergebnisse dokumentiert und diskutiert, sowie im Kapitel 5 werden ein Ausblick auf weiterführende Arbeiten und Zusammenfassung gegeben.
%%%%%%%%%%%%%%%%%%%%%%%%%%%%%%%%%%%%%%%%%%%%%%%%%%%%%%%%%%%%%%%%%%%%%%%%%%%%%%%%%%%
\section{Verwandte Arbeiten}
\subsection{Hintergrundsubtraktion}
Bewegungserkennung spielt heutzutage eine wichtige Rolle in Überwachungssystem. Durch die Hintergrundsubtraktion wird die Bewegung von dem Hintergrund extrahiert. Es gibt verschiedene wissenschaftliche Arbeiten für diese Methode. Ein der Verfahren ist ``Adaptive Background Mixture Models'' \cite{784637}, das ist eine verbreitete Methode zur Segmentierung einer bewegter Regionen in Bildsequenz. Die Methode beinhaltet eine ``Hintergrundsubtraktion'' oder eine Schwellenwertbindung des Unterschiedes zwischen einer Schätzung des Bild ohne sich bewegende Objekte und dem aktuellen Bild. In \cite{784637} wird das Modellieren jedes Pixels als eine Mischung von Gauß-Werten und das Verwenden einer Approximation zum Aktualisieren des Modells erläutert. Die Gaußschen Verteilungen des adaptiven Mischungsmodells werden dann ausgewertet, um zu bestimmen, welche am wahrscheinlichsten aus einem Hintergrundprozess resultieren.
\\
In \cite{elgammal2000non} wird eine Methode, die ``Kernel Density Estimation'' (\acs{KDE}) heißt, besprochen. \acs{KDE} ist ein neuartiges nicht-parametrisches Hintergrundsubtraktion. Das Modell kann mit Situationen umgehen, in denen der Hintergrund der Szene überladen und nicht vollständig statisch ist, sondern kleine Bewegungen wie Äste und Büsche enthält. Das Modell schätzt die Wahrscheinlichkeit der Intensitäten von Pixel auf der Grundlage einer Stichprobe von Intensitäten für jeden Pixel. Das Modell passt sich schnell an Veränderungen in der Szene an, was eine sehr empfindliche Erkennung bewegter Ziele ermöglicht.
\\
Eine verbesserte Version von \acs{KDE} ist in \cite{zivkovic2006efficient} als ``KNN"\ Methode beschrieben. \acs{KDE} benutzt eine feste Kerngröße $D$ für die gesamte Dichtefunktion, deswegen liefert es nicht immer beste Ergebnisse zurück. Diese Verbesserung passt die Kerngröße an jeden Pixel als Schätzpunkt an. Die konkrete Beschreibung befindet sich in nächstem Kapitel.
\\
In \cite{barnich2009vibe} und \cite{barnich2011vibe} wird ``Vibe'', eine leistungsfähige Methode zur Hintergrundextraktion vorgestellt. Die Hauptinnovation betrifft die Verwendung einer zufälligen Strategie zur Auswahl von Werten, um eine stichprobenartige Schätzung des Hintergrunds zu erstellen. Für jeden Pixel wird ein Satz von seinen Werten gespeichert, die in Vergangenheit am selben Ort oder in der Nachbarschaft genommen wurden. Dann wird dieser Satz mit dem aktuellen Pixelwert vergleichen, um zu bestimmen, ob dieser Pixel zu dem Hintergrund gehört. Das Modell von Vibe wird angepasst, indem der Satz zufällt ausgewählt wird, welche Werte aus dem Hintergrundmodell zu ersetzen sind. Es war das erste Mal, dass eine zufällige Wahrscheinlichkeit im Bereich der Hintergrundextraktion verwendet wird.
\\
Für diese Arbeit werden alle oben genannten Methoden verglichen, um eine vernünftige Methode herauszufinden. Die Vergleichungen aller oben genannten Methoden werden im nächsten Kapitel genauer beschrieben.\\
\subsection{Körperhaltungserkennung basiert auf Histogrammanalyse}
Histogrammanalyse ist eine einfache und schnelle Methode zur Erkennung der Körperhaltung. Es gibt viele wissenschaftliche Arbeite, die über Histogrammanalyse studiert und präsentiert werden. In\cite{guo2006projection} wird das Projektionshistogramm verwendet, um unterschiedliche Körperhaltungen zu unterscheiden. Ein neues Verfahren wird vorgeschlagen, dass ein Projektionshistogramm zur statischen Erkennung der menschlichen Haltung verwendet. Diese Arbeit in  \cite{guo2006projection} besteht aus drei Schlüsselmodulen: Hintergrundsubtraktion, Projektionshistogrammberechnung und Vorlagenabgleich.
\\
Eine andere Methode ist in \cite{haritaoglu1998ghost} beschrieben, die als ``Ghost"\ System benannt. ``Ghost"\ ist ein Echtzeitsystem zum Schätzen der menschlichen Körperhaltung und zum Detektieren von Körperteilen. Es erstellt ein Silhouetten-basiertes Körpermodell, um die Position der Körperteile zu bestimmen, während sich Personen in generischen Positionen befinden. Es kombiniert eine hierarchische Schätzung der Körperposen mit Histogrammanalyse, eine konvexe Rumpfanalyse der Silhouette und eine teilweise Abbildung von den Körperteilen zu den Silhouetten-Segmenten. Die Histogrammanalyse in dieser Masterarbeit basiert auf die Methode, die in ``Ghost"\ beschrieben wird.

 
