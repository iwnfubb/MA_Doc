% das Papierformat zuerst
%
%\documentclass[a4paper, 11pt,bibtotoc]{scrartcl}
\documentclass[a4paper, 11pt,bibtotoc,abstracton]{scrreprt}  

%F�r URL
\usepackage{url}
\renewcommand{\UrlFont}{\rmfamily}

%F�r Zitaten
\usepackage{cite}

%Abk�rzungen
\usepackage{acronym}


%Inhaltsverzeichnis bearbeiten
\usepackage{tocbibind}

% f�r mathematische Symbole
\usepackage{amsmath}


% Vektorgrafiken mit Latex importieren
\usepackage{import}

\usepackage[colorlinks=false, pdfborder={0 0 0}]{hyperref}
%documentclass[a4paper, 12pt]{article}  
% deutsche Silbentrennung
\usepackage[english,ngerman]{babel}
\renewcommand{\sectfont}{\rmfamily\bfseries}
% wegen deutschen Umlauten
\usepackage[ansinew]{inputenc}
\usepackage{graphicx}
\usepackage{subfigure}\hyphenation{Bit-rate}

%Algorithm schreiben
\usepackage{algorithm2e}

%Tabellen
\usepackage{booktabs}
\usepackage{multirow}
\usepackage{colortbl}


%Farben
\usepackage{color}
\usepackage{listings}%Code einbinden
\definecolor{darkblue}{rgb}{.08,.21,.36}
\definecolor{darkred}{rgb}{.6,.19,.20}
\definecolor{darkgreen}{rgb}{0,.6,0}
\definecolor{red}{rgb}{.98,0,0}
\definecolor{lightblue}{rgb}{0.8,0.85,1}
%\definecolor{lightgrey}{rgb}{0.98,0.98,0.98}
\definecolor{lightgrey}{gray}{.98}
\definecolor{black}{rgb}{0.0,0.0,0.0}


\lstloadlanguages{C++}
\lstset{%
  language=C++,
  basicstyle=\small,
  commentstyle=\itshape\color{darkgreen},
  keywordstyle=\bfseries\color{darkblue},
  stringstyle=\color{darkred},
  showspaces=false,
  showtabs=false,
  columns=fixed,
  backgroundcolor=\color{lightgrey},
  numbers=left,
  frame=single,
  numberstyle=\tiny,
  breaklines=true,
  showstringspaces=false,
  xleftmargin=1cm,
  basicstyle=\small
}%

\usepackage{amssymb}%Mathematische Symbole, wie R,N,Q,Z,...
\setlength{\parindent}{0pt} %einr�cken nach absatz verhindern
%\usepackage{setspace}%Zeilenabstand
\usepackage{algorithmic}%F�r Pseudocode

%%%%%%%%%%%%%%%%%%%%%%%%%%%%%%%%%%%
%Seiten Kopf- und Fu�zeilen
\usepackage[automark,						
		headsepline,								
		plainfootsepline, 
		]{scrpage2}

\automark[section]{chapter} 
\pagestyle{scrheadings}			

\clearscrheadings	%Alte Kopfformatierungen entfernen
\clearscrplain		%Alte Plain-Formatierung entfernen
\clearscrheadfoot %Alten Fu� entfernen
\cfoot[\pagemark]{\pagemark}%Seitenzahl zentriert im Fu� 
\ihead{\leftmark}
\ohead{\rightmark} 
 
%%%%%%%%%%%%%%%%%%%%%%%%%%%%%%%%%%%

\begin{document}
\begin{titlepage}
\begin{center}
{\huge \textbf{Philipps-Universit�t Marburg}}\\[0.5cm]
\textbf{Fachbereich 12 - Mathematik und Informatik}\\[0.5cm]

\begin{figure}[h]
	\centering
		\includegraphics[width=0.8\textwidth]{fig/unilogo.pdf}
\end{figure}

{\huge \textbf{{\large \\[1cm]Diplomarbeit}}}
\\[1cm]

{\Huge \textbf{Progressive Meshes\\ mit CUDA}}
\\[1cm]

{\large von}\\
{\large Max Mustermann}\\
{\large Dezember 2009}\\[3cm]

{\large
Betreuer:\\ Prof. Dr. Thorsten Thorm�hlen\\[1cm]
Arbeitsgruppe Grafik und Multimedia Programmierung}

\end{center}
\end{titlepage}
\newpage
\thispagestyle{empty}
\section*{}
\newpage
\thispagestyle{empty}
\vspace*{7cm}
\textbf{\Large {Erkl�rung}}\\[0.5cm]
Ich, Max Mustermann (Informatikstudent an der Philipps-Universit�t Marburg, Matrikelnummer
xxxxxx), versichere an Eides statt, dass ich die vorliegende Diplomarbeit selbstst�ndig verfasst und keine anderen
als die angegebenen Quellen und Hilfsmittel verwendet habe. Die hier vorliegende Diplomarbeit wurde weder in ihrer jetzigen noch in einer �hnlichen Form einer Pr�fungskommission vorgelegt.\\[1cm]
Marburg, 17. Dezember 2009\\[0.5cm]
Max Mustermann
\newpage
\shipout\null

%\setstretch {1.15}%Zeilenabstand setzen

  
% hier beginnt das Dokument